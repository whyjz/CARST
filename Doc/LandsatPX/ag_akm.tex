\documentclass[12pt]{article}
\usepackage{graphicx}
\usepackage{longtable}
\usepackage{lineno}
\usepackage{natbib}
\usepackage[margin=1in]{geometry}

\newenvironment{myindentpar}[1]%
{\begin{list}{}%
 {\setlength{\leftmargin}{#1}}%
  \item[]%
 }
{\end{list}}

\title{Preparing Elevation Data Using ``aggregate.py''}

\author{Andrew Kenneth Melkonian}
\begin{document}

\maketitle

\vspace{2pt}
\hrule
\vspace{6pt}
\noindent {\bf aggregate.py}
\vspace{6pt}
\hrule
\vspace{16pt}

\noindent {\bf \underline{Description}:} \\

\noindent ``aggregate.py'' reads five-column ASCII text files (longitude, latitude, elevation, date in decimal years, uncertainty), with each file corresponding to a DEM, with the path to the reference DEM specified separately.
It creates five-column ASCII text files as output, with the columns the same as the input DEMs.
The data in the output files is organized on a pixel-by-pixel basis, with pixels separated by a single line containing the greater than character (``$>$'', e.g., GMT polygon file format).
This script prepares the input for the ``weightedRegression.py'' script. \\

\noindent {\bf \underline{Dependencies}:} \\

\noindent {\bf Python} \\

\noindent {\bf \underline{Usage}:} \\

\noindent python aggregate.py reference\_dem\_txt\_path input\_dem\_txts\_directory identifier increments \\

\noindent {\bf \underline{Example}:} \\

\noindent python /path/to/ref\_dem.txt /path/to/inputs/ app 5 \\

\noindent {\bf \underline{Input Parameters}:} \\

\noindent {\bf reference\_dem\_txt\_path:} Path to 5-column ASCII text file of longitudes, latitudes, elevations, date in decimal years, and uncertainties for reference DEM. \\

\noindent {\bf input\_dem\_txts\_directory:} Path to directory containing input DEMs. The ASCII text file for each input DEM must have the same format as the ASCII text file for the reference DEM. \\

\noindent {\bf identifier:} String value, input DEMs in input\_dem\_txts\_directory must end in "identifier.txt" to be read in. \\

\noindent {\bf increments:} Must be a number; the output will be split into this many files (e.g., "glacier\_ice\_values\_1.txt", "glacier\_ice\_values\_2.txt" if increments is "2"). \\

\noindent {\bf \underline{Output}:} \\

\noindent ASCII text file(s) with all of the data organized by pixel (pixels separated by single lines containing ``$>$'' character).
The "increments" input parameter determines how many of these will be made.
Each file has five columns: \\
1) Longitude (floating point) \\
2) Latitude (floating point) \\
3) Elevation (floating point) \\
4) Decimal year (floating point) \\
5) Uncertainty (floating point) \\



\end{document}
